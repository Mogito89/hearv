% $File: method.tex
% $Date: Sat Nov 29 16:42:47 2014 +0800
% $Author: jiakai <jia.kai66@gmail.com>

\section{方法}

总体研究思路比较直接:基于对图像微小运动的分析得到物体震动的情况,
再对应回原始声音;为了在简陋的实验条件下提高采样率,要利用卷帘快门的特性。
详述如下:

\subsection{卷帘快门建模及参数估计\label{sec:rolling-shutter}}
% f{{{
数码相机的快门一般有全局快门和卷帘快门两大类。对于卷帘快门,快门速度为$E$时,
其第$y$行对应的曝光时段近似为:
\begin{equation}
    I_y = [yd, yd+E]
\end{equation}
其中$d$是线延迟,也就是感光器的两行像素间曝光的时间差。这样,
如果把每行看作单独的一帧,我们相当于达到了$1/d$的采样率。

\cite{Davis2014VisualMic}采取测量直线斜率的方法得到Pentax K-01
相机的线延迟为16微秒,但没有详述其具体实现。我们缺乏生成高速运动直线的设备,
因此想了自己的测量方法。

对于具有亮度调节功能的LED灯,其实现低亮度的方法往往是降低光源信号的占空比,
因此相当于有一个高速频闪的光源。在较高快门下对这样的光源录影,可得到条纹图像,
如\figref{rolling-shutter-record}所示。
\begin{figure}[tb]\begin{center}
    \begin{subfigure}[b]{.5\figwidth}
        \centering
        \includegraphics[width=.5\figwidth]{res/rolling0.png}
        \caption{1/1000s 快门速度}
    \end{subfigure}
    \begin{subfigure}[b]{.5\figwidth}
        \centering
        \includegraphics[width=.5\figwidth]{res/rolling1.png}
        \caption{1/4000s 快门速度}
    \end{subfigure}
    \caption{用卷帘快门对高速频闪光源下的坐标纸成像结果;根据快门速度和亮条高度
        的关系可计算出线延迟。\label{fig:rolling-shutter-record}}
\end{center}\end{figure}

设$[T_0, T_1]$时段里光源点亮,则此时能呈现亮条的行$y$需满足条件:
\begin{equation}
    [yd, yd+E] \cap [T_0, T_1] \ge T_\theta
\end{equation}
其中$T_\theta$为需要点亮一行像素所需的最短曝光时间。如果光源点亮时长恒定,
当$E$变化时,亮条高度也会变化,有如下关系:
\begin{equation}
    \Delta_H d \approx \Delta_E
\end{equation}

因此通过测量亮条高度和快门速度间的关系,可以推出线延迟。
我们购买了与\cite{Davis2014VisualMic}中同样型号的相机,
测量七组数据求得的平均线延迟为18.3微秒,与原文中的结论较为接近,
对应的采样率约为54644Hz。
% f}}}

\subsection{运动分析:基于Complex Steerable Pyramid分解}
% f{{{
\cite{Davis2014VisualMic}采取了基于Complex Steerable Pyramid的方法进行运动恢复,
在此我们先对其进行简单叙述。

Complex Steerable Pyramid是可以把图片分解为不同方向和尺度的子频带滤波器组,
最初由Portilla和Simoncelli提出用于纹理分析和合成\cite{Portilla99}。
对于单通道图像$I: \mathbb{Z}^2 \mapsto \mathbb{R}$和给定的尺度$r$及旋转方向
$\theta$,在$I$的每一个局部$(x, y)$附近,可利用Complex Steerable Pyramid
将其对应的频带表示为:
\begin{equation}
    A(r, \theta, x, y) e^{i\varphi(r, \theta, x, y)}
\end{equation}
扩展到视频中,可得到$t$时刻时某空间位置对应的相位$\varphi(r, \theta, x, y, t)$。
则相对某参考帧$t_0$的相位差
\begin{equation}
    \varphi_v(r, \theta, x, y, t) = \varphi(r, \theta, x, y, t) - 
    \varphi(r, \theta, x, y, t_0)
\end{equation}
对应于$t$时刻$(x, y)$点在$(r, \theta)$方向上的运动。

有了局部运动信息后,对每个$(r_i, \theta_i)$二元组计算全局平均运动:
\begin{equation}
    \Phi(r_i, \theta_i, t) = \sum_{x, y}
    A(r, \theta, x, y, t)^2 \varphi_v(r, \theta, x, y, t)
\end{equation}

随后,在时间上对$\Phi(r_i, \theta_i, t)$进行平移以防止不同方向的震动相抵消:
\begin{equation}
    t_i = \arg\max_{t_i} \trans{\Phi(r_0, \theta_0, t)}\Phi(r_i, \theta_i, t +
    t_i)
\end{equation}

最后,对所有$\Phi(r_i, \theta_i, t + t_i)$按$i$求平均就是$t$时刻全局运动,
对应此时的声音信号。

但该方法实现比较复杂而且计算量较大,我们目前还没测试,而是使用了下文所述的基于
Riesz变换的方法。

% f}}}

\subsection{运动分析:基于Riesz变换}
% f{{{
Riesz变换\cite{felsberg2001monogenic}是对分析信号的二维扩展。
一个一维实信号的分析信号是复信号,由原信号加上其希尔伯特变换作为虚部得到,
由分析信号的相位差可以检测出两个实信号的平移变化。Riesz变换将此扩展到二维,
可以把二维实信号分解为振幅、幅角、相位三个分量,对于图像而言,
相位变化对应原图在空域上的位移。Unser等人将此扩展到了多分辨率
\cite{unser2009multiresolution},而Wadhwa等则实现了在空域上操作的近似Riesz变换
并以此实现实时运动放大\cite{Wadhwa2014RieszPyramid}。

具体而言,Riesz变换由一对滤波器组成,其转移函数分别为
\begin{equation}
    -i\frac{\omega_x}{\parallel \overrightarrow{ \omega} \parallel},~
    -i\frac{\omega_y}{\parallel \overrightarrow{ \omega} \parallel}
\end{equation}
将其应用到子频带图像$I$上时,得到滤波器响应$(R_1, R_2)$,于是局部的振幅$A$、
主方向$\theta$和相位$\phi$满足:
\begin{equation}
    I = A\cos(\phi),~R_1 = A\sin(\phi)\cos(\theta),~
    R_2 = A\sin(\phi)\sin(\theta)
    \label{eqn:riesz:decomp}
\end{equation}

而\cite{Wadhwa2014RieszPyramid}中,作者证明了用$[0.5,0,-0.5]$和
$\trans{[0.5,0,-0.5]}$两个空域上的卷积核就可以较好地近似Riesz变换。

但Riesz变换尚未被用于全局运动分析,在\cite{Wadhwa2014RieszPyramid}中也仅用于局
部运动分析来进行运动放大。如果要用于全局分析,有个本质性的问题是
$(A, \phi, \theta)$和$(A, -\phi, \theta + \pi)$都是\eqnref{riesz:decomp}的解,
无法唯一确定相位,也就无法简单地求加权平均来得到全局平均相位。

为了解决这个问题,我们提出了朝向平滑算法。对于图像$(A(x, y), \phi(x, y),
\theta(x, y))$,设其平滑后的图像为$(A_s(x, y), \phi_s(x, y), \theta_s(x, y))$,
需要求出$k(x, y)\in \mathbb{Z}$,满足:
\begin{eqnarray}
    |\theta(x, y) + k(x, y)\pi - \theta_0(x, y)| &\le& \frac{\pi}{2} \nonumber \\
    A_s(x, y) &=& A(x, y) \nonumber \\
    \theta_s(x, y) &=& \theta(x, y) + k(x, y)\pi \nonumber \\
    \phi_s(x, y) &=& (-1)^k\phi(x, y)
\end{eqnarray}
其中$\theta_0(x, y)$为目标朝向,先各行独立进行指数滤波平滑,
$\theta_0(x+1, y) = \alpha\theta_0(x, y) + (1-\alpha)\theta_s(x, y)$,
再同理在各行之间进行平滑。平滑算法的效果如\figref{smooth}所示。
\begin{figure}[tb]\begin{center}
    \begin{subfigure}[b]{.5\figwidth}
        \centering
        \includegraphics[width=.5\figwidth]{res/smooth-0.png}
        \caption{未进行朝向平滑}
    \end{subfigure}
    \begin{subfigure}[b]{.5\figwidth}
        \centering
        \includegraphics[width=.5\figwidth]{res/smooth-1.png}
        \caption{进行朝向平滑}
    \end{subfigure}
    \caption{朝向平滑算法效果\label{fig:smooth}}
\end{center}\end{figure}

对于两帧$F_1 = (A_1, \phi_1, \theta_1)$和$F_2 = (A_2, \phi_2, \theta_2)$,
先将$F_1$按上述方法进行朝向平滑得到$F_{1s}$,再令上述算法中的
$\theta_0=\theta_{1s}$对$F_2$平滑得到$F_{2s}$,使得$\theta_{2s}$与$\theta_{1s}$
尽量接近,于是可得全局相位差
\begin{equation}
    \delta_\phi(F_1, F_2) = \sum_{x, y}
    \left(\frac{A_{1s}(x, y) + A_{2s}(x, y)}{2}\right)^2(\phi_{2s}(x, y) -
    \phi_{1s}(x, y))
\end{equation}

% f}}}

\subsection{创新之处}
到目前为止,有两个比较创新的地方:
\begin{enumerate}
    \item 仅利用频率未知的高频闪烁LED光源观察到卷帘快门的效果,
        并估算出线延迟。
    \item 利用Riesz变换进行全局运动分析,并提出了空间朝向平滑的算法。
\end{enumerate}

% vim: filetype=tex foldmethod=marker foldmarker=f{{{,f}}} 

