% $File: discuss.tex
% $Date: Wed Jan 07 20:53:05 2015 +0800
% $Author: jiakai <jia.kai66@gmail.com>

\section{分析与讨论}
通过在合成数据上的实验(\secref{synth-global}、\secref{synth-local}),
可以发现在理想情况下,
基于Riesz变换的局部运动分析的方法在恢复低频全局运动和高频逐行局部运动上
均是比较有效的,在噪声存在的情况下对于$0.01$像素振幅的信号仍能进行较好的重建。

在实际拍摄的视频上,我们发现对单频音能较有效的恢复。然而,
经过多次尝试,仍不能取得满意的双频音恢复效果,
也仅仅是能在\figref{real:300+500}中在500Hz附近看到一个小尖峰。
对此我们尚不能给出一个很好的解释,猜想是对实际信号中振幅相对大小恢复的不够好,
导致低频信号盖过高频信号。对比\figref{synth:500+800}的结果,
也说明我们合成的数据相对实际数据而言还是太过于理想了。

总的来说,在本文中,
我们介绍了基于Reisz变换从卷帘快门摄得的单帧图像恢复高频震动,
并对视频各帧重建出的信号进行整体优化从而得到人耳可辨识的音频信号的方法,
在合成数据上能取得较好效果,在实际数据上也成功恢复了单频简单音乐。 
当然,本工作也具有一些局限性:
\begin{enumerate}
    \item 单帧内频谱分析所对应的时域信号总时长只有$Hd \approx 0.01$s,
        因此频谱的谱线间距约为$100$Hz,频谱解析度较差。
    \item 对高频信号的恢复丢失了相位信息,虽然对音频的听觉效果影响不大,
        但毕竟有信息损失。
    \item 对于多频音尚无法有效恢复。
    \item 在音源音量较小,也就是震动不明显时,几乎无法恢复出有用的信息,
        这应该是因为视频本身的信噪比就很低:被拍摄物体完全静止的两帧,
        像素差的RMS为$2.08$,而在播放声音时该值也仅为$2.95$。
        这噪声来自多方面,
        比如低端相机本身的系统噪声和H.264视频编码引入的artifact。
\end{enumerate}

% vim: filetype=tex foldmethod=marker foldmarker=f{{{,f}}} 
