% $File: discuss.tex
% $Date: Sun Dec 14 00:33:39 2014 +0800
% $Author: jiakai <jia.kai66@gmail.com>

\section{分析与讨论}
通过在合成数据上的实验(\secref{synth-global}、\secref{synth-local}),
可以发现在理想情况下,
基于Riesz变换的局部运动分析的方法在恢复低频全局运动和高频逐行局部运动上
均是比较有效的,在噪声存在的情况下对于$0.01$像素的运动仍能给出明确响应。

在实际拍摄的视频上,我们发现对单频音仍能较有效的恢复。然而,
经过多次尝试,仍不能取得满意的双频音恢复效果,
也仅仅是能在\figref{real:300+500}中在500Hz附近看到一个小尖峰。
对此我们尚不能给出一个很好的解释,猜想是对实际信号中振幅相对大小恢复的不够好,
导致低频信号盖过高频信号。对比\figref{synth:500+800}的结果,
也说明我们合成的数据相对实际数据而言还是太过于理想了。

在此,先提出本工作的一些{\bf 局限性}:
\begin{enumerate}
    \item 单帧内频谱分析所对应的时域信号总时长只有$Hd \approx 0.01$s,
        因此频谱的谱线间距约为$100$Hz,频谱解析度较差。
    \item 对高频信号的恢复丢失了相位信息,虽然对音频的听觉效果影响不大,
        但毕竟有信息损失。
    \item 对于多频音尚无法有效恢复。
    \item 在音源音量较小,也就是震动不明显时,几乎无法恢复出有用的信息;
        而且即使被摄物体完全静止,也会较大噪声。这噪声来自多方面,
        但主要应该是低端相机本身的系统噪声和H.264视频编码引入的artifact。
\end{enumerate}

{\bf 下一步}工作是重建声音信号。在不考虑目前仅能有效恢复单频音的情况下,
希望利用频谱上的全部信息(而非直接取最大能量的频率作为重建结果)。
对此已进行了不少尝试,主要的困难有两个:
(1) 卷帘快门的总曝光时间少于帧时长,导致部分时间段内没有采样;
(2) 各帧的能量不统一,恢复出来的声音不平滑,听起来有强烈的低频节奏
而很难听到高频的原始信号。
接下来需要在降噪和全局平滑上进行更多努力,具体细节在以后的报告中详述。

与已有工作相比,本工作的{\bf 创新点}主要如下:
\begin{enumerate}
    \item 仅利用频率未知的高频闪烁LED光源观察到卷帘快门的效果,
        并估算出线延迟。
    \item 利用Riesz变换进行全局和局部的运动分析,并提出了空间朝向平滑的算法。
        Riesz变换可以直接在空域上进行,
        其运行速度也优于Complex Steerable Pyramid。
    \item 基于信号处理的基本原理,提出了从卷帘快门图像恢复高频震动的算法,
        这在已有的工作中也并没有专门讨论过。
\end{enumerate}

\section{后续工作}
在一到两周内实现能达到人耳可接受音质的单频音简单旋律的恢复,
随后看具体情况可能在以下一种或几种方向上尝试:
\begin{enumerate}
    \item 拍摄实际物体的震动而非扬声器表面的震动,
        尝试对声音的恢复
    \item 实现Complex Steerable Pyramid并进行效果的对比
    \item 设法解决多频音恢复的问题,尝试对更复杂的声音,包括语音、音乐等的恢复
\end{enumerate}

% vim: filetype=tex foldmethod=marker foldmarker=f{{{,f}}} 
