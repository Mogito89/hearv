% $File: misc.tex
% $Date: Sat Nov 29 16:41:56 2014 +0800
% $Author: jiakai <jia.kai66@gmail.com>

\section{进度计划}

\subsection{中期检查}
预计可完成初步的单频音的高频信号恢复,并能听到声音;
重要需要解决一些全局平滑的问题。

\subsection{期末检查}

预计能完成基于直接拍摄扬声器的图像的多频音高频信号恢复,
主要难点在降噪和对帧间无采样区间的插值。
希望能够恢复出可接受的音乐或语音。
另外最终算法确定后会进行一些工程上的加速,希望能做到实时
(\cite{Davis2014VisualMic}中处理数秒视频需要几个小时)。

还不能确定能否实现对其它物体表面震动的被动恢复,
因为感觉拍摄的噪声比较大,而且H.264编码会引入时间上的artifact,
并且恢复效果也与音量、物体材质、音源方向等有关,
需要进一步大量实验。


\section{资料文档结构及说明}
所有代码均在\url{https://github.com/jia-kai/hearv}上,目录结构描述如下:

\dirtree{%
    .1 /.
    .2 camera\_test\DTcomment{生成规律性图案测试相机,后来没用上}.
    .2 disp\_freq.py\DTcomment{显示一维数组及其FFT}.
    .2 extract\_img.sh\DTcomment{从视频中抽取指定时间段的图片文件}.
    .2 get\_rolling.py\DTcomment{自动分析卷帘快门的图像并计算线延迟}.
    .2 playground\DTcomment{尝试用opencv的光流和phase correlation计算运动的脚本}.
    .2 riesz.
    .3 riesz.py\DTcomment{Riesz变换及后处理}.
    .3 fastmath.py\DTcomment{cython代码,目前仅有朝向平滑算法的实现}.
    .2 data\DTcomment{3.2G的各种实验视频和图片文件}.
    .2 midterm\DTcomment{本中期报告的源码}.
}

% vim: filetype=tex foldmethod=marker foldmarker=f{{{,f}}} 
